\documentclass[11pt]{amsart}
\usepackage{geometry}                % See geometry.pdf to learn the layout options. There are lots.
\geometry{a4paper}                   % ... or a4paper or a5paper or ... 
%\geometry{landscape}                % Activate for for rotated page geometry
\usepackage[parfill]{parskip}    % Activate to begin paragraphs with an empty line rather than an indent
\usepackage{graphicx}
\usepackage{amssymb}
\usepackage{epstopdf}
\usepackage{cite}
\usepackage{hyperref}
\DeclareGraphicsRule{.tif}{png}{.png}{`convert #1 `dirname #1`/`basename #1 .tif`.png}

\title{EKFmonoSLAM for the IntelliTable robot}
\author{Harry Jackson}
%\date{}                                           % Activate to display a given date or no date

\begin{document}
\maketitle
%\tableofcontents
\section{Introduction}
The EKFmonoSLAM uses a EKF SLAM algorithm originally designed by Joan Sola ~\cite{sola2012impact}.
\section{Quick Set-up}
This section outlines the steps that need to be taken to set up the SLAM algorithm software in MATLAB. This set-up assumes you have MATLAB already installed.
\\
The software for running the SLAM algorithm is publicly available online\footnote{\url{https://github.com/HJackson3/intelliTable-SLAM}}.

\section{Analysis}

\subsection{Processing speed considerations}

\subsection{Average mahalanobis distance}

\section{Transfer to the intelliTable}

\section{Next steps}
At this current stage, the algorithm has been adapted to work with a camera attached to the robot, but the robot's navigation needs to be implemented as well. 

\clearpage
\bibliography{myBib}{}
\bibliographystyle{plain}
\end{document}