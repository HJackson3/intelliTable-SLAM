\documentclass[1pt, oneside]{article}
\usepackage{geometry}                % See geometry.pdf to learn the layout options. There are lots.
\geometry{a4paper}                   % ... or a4paper or a5paper or ... 
%\geometry{landscape}                % Activate for for rotated page geometry
\usepackage[parfill]{parskip}   	 % Activate to begin paragraphs with an empty line rather than an indent
\usepackage{graphicx}
\usepackage{amssymb}
\usepackage{epstopdf}
\usepackage{cite}
\usepackage{hyperref}
\DeclareGraphicsRule{.tif}{png}{.png}{`convert #1 `dirname #1`/`basename #1 .tif`.png}
%\bibliographystyle{plain}

\title{EKFmonoSLAM for the IntelliTable robot}
\author{Harry Jackson
	\\University of Sheffield
	\\
	\texttt{hjackson3@sheffield.ac.uk}
	}
%\date{\today}                                           % Activate to display a given date or no date

\begin{document}
\maketitle
\tableofcontents
\clearpage
\section{Introduction}
The EKFmonoSLAM uses a EKF-SLAM algorithm originally designed by Joan Sola\cite{sola2012impact, sola2008fusing}.
\section{Quick Set-up}
This section outlines the steps that need to be taken to set up the SLAM algorithm software in MATLAB. This set-up assumes you have MATLAB already installed.
\\

To run the toolbox using an actual robot, the Robotics System Toolbox is required. The software for running the toolbox is publicly available online\footnote{\url{https://github.com/HJackson3/intelliTable-SLAM}}. The repository contains detailed instructions on set-up.
\\

The software here is designed for use with a Kuka Youbot but the core functionality should work with any robot.

\section{Analysis}

\subsection{Processing speed considerations}
The average processing time is roughly 0.2s. This means that the SLAM algorithm is currently working at 5Hz on a Intel 3.5GHz processor. This is not as fast as other visual SLAMs\cite{davison2003real}, but is fast enough for purpose as the intelliTable will not move at particularly high speeds.
\\

The slower processing speed has some advantages though; as the camera frame needs to be called on every iteration, only calling this 5 times per second means that it is unlikely that there will be lag between calling and receiving the image which might affect the ability of the SLAM.

\subsection{Average mahalanobis distance}

\section{Next steps}
At this current stage, the algorithm has been adapted to work with a camera attached to the Youbot, but path planning is still not put in place. 

\subsubsection*{Navigation}
The robot can move very simply - in a straight line until possible collision, then rotates clockwise until it would no longer collide - but no other navigation has been implemented yet.

\subsection{Transfer to the intelliTable}

\subsection{Possible improvements}

\clearpage
\bibliographystyle{plain}
\bibliography{myBib}{}
\end{document}